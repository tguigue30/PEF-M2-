{\centering
\vspace*{\fill}
\section*{Résumé}}
\addcontentsline{toc}{section}{Résumé}
\begin{sloppypar}
  Ce projet vise à modéliser les dynamiques des sous-indicateurs de stress systémique des marchés interbancaire, des changes et des actions à l’aide de données mensuelles de 2005 à 2023. Ces marchés, interdépendants et soumis à des tensions systémiques, sont étudiés à travers des approches économétriques non linéaires, notamment les modèles Markov Switching VAR et NARDL. Ces outils permettent de capturer les transitions entre régimes et les asymétries dans les réponses aux chocs financiers. En s’appuyant sur le Composite Indicator of Systemic Stress, le projet analyse les interconnexions entre ces marchés pour identifier les mécanismes de contagion et les périodes de forte instabilité. L’objectif est d’approfondir la compréhension des tensions systémiques et de fournir des outils pour anticiper les crises financières, renforcer la stabilité des marchés et guider les politiques économiques et macroprudentielles. 
\end{sloppypar}
\vspace*{\fill}