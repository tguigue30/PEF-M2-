\section*{Conclusion}
\addcontentsline{toc}{section}{Conclusion}

Ce projet d’économétrie financière visait à analyser et modéliser les dynamiques non linéaires des sous-indicateurs de stress systémique des marchés interbancaire, des changes et des actions sur la période 2005-2023 (sur données mensuelles). En intégrant des approches économétriques avancées, notamment les modèles MS-VAR et NARDL, il a été possible d'identifier les interconnexions et mécanismes de contagion entre ces marchés, tout en prenant en compte les effets de régime et les asymétries qui caractérisent leurs interactions.\\

En ce qui concerne le marché actions, il a présenté le niveau de stress systémique moyen le plus élevé (8.02) et la volatilité la plus marquée parmi les trois segments étudiés. Les crises de 2008 et 2020 ont été des périodes particulièrement stressantes, marquées par des pics significatifs de volatilité. Ces résultats montrent que le marché des actions est le plus sensible aux événements macroéconomiques et politiques, avec une grande réactivité aux changements des conditions du marché mondial.\\

Pour le marché interbancaire, bien que plus stable globalement, ce marché a montré des épisodes de stress extrême, comme en 2008, avec une forte asymétrie positive (skewness de 1.45) et une kurtosis élevée (5.35). Ces caractéristiques reflètent des tensions aiguës mais transitoires, souvent maîtrisées par les interventions des banques centrales.\\

Enfin, pour le marché de changes, il est constaté la plus faible volatilité relative, avec un stress moyen de 4.19. Cependant, des épisodes de stress ponctuels, notamment durant les crises financières et géopolitiques, ont révélé son rôle clé dans la transmission des chocs internationaux.\\

Le résultat le plus intéressant de cette étude concerne l'identification des régimes. Les résultats des modèles MS-VAR ont permis d’identifier deux régimes principaux : un régime de faible stress, caractérisé par une volatilité élevée et une faible corrélation entre les sous-indicateurs, et un régime de fort stress, marqué par une volatilité basse et une corrélation élevée. À première vue, cette découverte peut paraître contre-intuitive : il serait en effet plus logique qu’un régime de fort stress s’accompagne d’une volatilité accrue et que, dans un contexte de faible stress, les corrélations entre les marchés soient moindres. Toutefois, plusieurs explications furent apportées.\\

La volatilité élevée en régime de faible stress comme un reflet de l’activité de marché. En effet, en période de faible stress, les marchés fonctionnent normalement avec une activité dynamique et des ajustements constants des prix. Cette activité accrue génère une volatilité élevée sur le stress, reflétant des fluctuations fréquentes mais modérées, en lien avec des facteurs économiques habituels. Cependant, ces fluctuations ne sont pas fortement synchronisées entre les segments de stress (faible corrélation), car chaque marché réagit principalement à ses propres déterminants fondamentaux.\\

La corrélation élevée en régime de fort stress est un effet de synchronisation en période de crise. En période de fort stress, les marchés tendent à se synchroniser en raison de mécanismes de contagion et de comportements coordonnés des investisseurs. Par exemple, lors d’une crise, les acteurs du marché se tournent massivement vers des actifs refuges (obligations souveraines, devises stables), amplifiant les interdépendances entre les segments financiers. Cela entraîne une corrélation élevée entre les sous-indicateurs, même si la volatilité des prix peut être réduite par une baisse générale de l’activité de marché (moins de transactions, liquidité en baisse). Les acteurs adoptent des comportements défensifs, ce qui stabilise temporairement les prix tout en maintenant des tensions systémiques élevées.\\

Baisse de la volatilité en régime de fort stress : un phénomène lié à la paralysie des marchés. Lorsque les tensions systémiques atteignent un niveau critique, les marchés peuvent entrer dans une phase de paralysie. La baisse de la liquidité, la réticence des investisseurs à prendre des risques et l’intervention des régulateurs contribuent à réduire les fluctuations des prix, d’où une volatilité plus faible en apparence. Cependant, cette stabilité apparente masque des tensions, comme le risque de faillites ou de ruptures de confiance.\\

En somme, il ne faut pas associer automatiquement la volatilité aux périodes de stress élevé. La dynamique des régimes identifiés montre l’importance d’examiner conjointement les niveaux de corrélation, de volatilité et de stress systémique pour comprendre la propagation du stress systémique.\\

Aussi, les transitions entre ces régimes sont souvent associées à des chocs exogènes majeurs, tels que les crises financières de 2008 et 2020. Ces transitions illustrent les non-linéarités dans les interactions entre les marchés : un choc initial sur un segment peut entraîner un passage soudain d’un régime à un autre, amplifiant les tensions systémiques. Les résultats montrent que les périodes de faible stress sont significativement plus longues que celles de fort stress, indiquant une certaine résilience des marchés en l’absence de perturbations majeures. En moyenne, les marchés restent dans le régime de faible stress pendant environ 12 mois, contre 3 à 4 mois dans le régime de fort stress.\\

L’analyse des interconnexions a révélé une corrélation significative entre les trois sous-indicateurs. Cette interdépendance montre la propagation des tensions entre les segments en période de crise, amplifiée par des comportements synchronisés des investisseurs et les ajustements des portefeuilles. Une hausse de la volatilité sur le marché des actions a souvent entraîné une pression accrue sur le marché des changes, via des ajustements dans les flux de capitaux internationaux. De même, les crises de liquidité sur le marché interbancaire ont exacerbé les tensions sur les autres segments, les banques étant contraintes de liquider leurs actifs.\\

Les mécanismes de contagion identifiés incluent la fuite vers la qualité (« flight-to-quality »), les arbitrages spéculatifs et la propagation via les liquidités. Ces mécanismes sont renforcés par des asymétries dans les réponses aux chocs. Par exemple, les chocs négatifs ont des effets plus marqués que les chocs positifs, notamment sur le marché des actions, ce qui souligne l'importance de différencier les dynamiques selon le signe des variations.\\

Pour le régulateur. Les résultats mettent en évidence le rôle crucial des politiques monétaires et macroprudentielles pour atténuer les tensions systémiques. Le CISS, utilisé comme outil de surveillance, permet de détecter les signaux précoces de crises potentielles et d’intervenir de manière préventive. Par exemple, des injections de liquidités ou des ajustements des exigences de fonds propres peuvent limiter la propagation des tensions.\\

Ce travail a démontré l’utilité des modèles non linéaires pour comprendre les dynamiques complexes des marchés financiers. L’intégration des effets de régimes et des asymétries a permis de mieux saisir les interactions entre les segments, fournissant ainsi des outils robustes pour guider les décisions des régulateurs.\\

En surveillant simultanément les segments clés du système financier, le CISS permet d’identifier les périodes critiques et d’intervenir rapidement pour éviter une contagion systémique.
Renforcement de la résilience : Les résultats appuient la nécessité de renforcer les exigences macroprudentielles, comme l’imposition de coussins de capital anticrise. Les asymétries observées suggèrent que les réponses aux chocs doivent être différenciées selon leur nature. \\

Les dynamiques complexes et non linéaires identifiées dans cette étude soulignent la nécessité d’aller au-delà des modèles économétriques traditionnels pour mieux comprendre les crises systémiques. Une piste réside dans l’utilisation des méthodes éconophysiques, inspirées de la physique statistique, pour modéliser les comportements des marchés financiers. Ces approches, basées sur des systèmes dynamiques non linéaires, permettent de capturer des phénomènes tels que les transitions de phase, les états critiques et les comportements émergents dans les réseaux financiers.\\

En particulier, le modèle $\phi^4$, issu de la physique des champs, offre des perspectives intéressantes pour analyser les interactions entre agents économiques dans des systèmes fortement couplés. Ce modèle, utilisé initialement pour étudier les transitions de phase dans les matériaux, peut être adapté pour représenter les changements soudains entre régimes de faible et de fort stress dans les marchés financiers. Par sa capacité à modéliser des états métastables et des bifurcations, le modèle $\phi^4$ pourrait compléter les approches économétriques en fournissant une vision plus structurelle des crises systémiques.\\

De plus, les méthodes éconophysiques permettent d'exploiter pleinement les réseaux financiers et les interconnexions cachées au sein des systèmes complexes. Couplées aux avancées en Big Data et en machine learning, ces approches pourraient améliorer la précision des prévisions.