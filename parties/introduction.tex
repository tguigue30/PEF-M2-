\section*{Introduction}\addcontentsline{toc}{section}{Introduction}

\begin{sloppypar}

La stabilité du système financier européen repose sur l’équilibre entre les différents segments de marché. Cependant, cet équilibre est régulièrement menacé par des tensions systémiques, amplifiées par l’interconnexion croissante des marchés et les mécanismes de contagion. Ces tensions, lorsqu’elles ne sont pas correctement identifiées, peuvent rapidement se propager et engendrer des crises financières globales, comme les crises des subprimes de 2008, la crise de la dette souveraine en Europe entre 2010 et 2012, et plus récemment la pandémie de COVID-19. Ces événements mettent en lumière la nécessité de disposer d’outils avancés pour comprendre et anticiper les dynamiques complexes qui sous-tendent les stress systémiques.\\

Dans ce contexte, le Composite Indicator of Systemic Stress (CISS) émerge comme un outil essentiel pour capter les signaux de stress financier. Ce dernier repose sur l’agrégation de cinq sous-indicateurs de stress spécifiques à trois segments financiers majeurs : le marché interbancaire, le marché des changes, le marché des actions, le marché actions et les intermédiaires financiers. Ces sous-indicateurs permettent de mesurer non seulement les tensions au sein de chaque marché, mais aussi les interactions dynamiques entre ces segments, particulièrement critiques en période de crise. Pourtant, la complexité des dynamiques de marché, exacerbée par de fortes volatilités avec des changements de régime et des effets asymétriques, rend leur modélisation particulièrement ardue.\\

En ce qui concerne la modélisation du stress systémique, des outils comme le CISS ont été développés pour fournir des mesures agrégées du stress financier. \cite[2012]{Hollo} ont conceptualisé le CISS, démontrant son utilité pour capturer les interconnexions et les corrélations dynamiques entre plusieurs segments financiers. D’autres travaux, tels que ceux de \cite[2013]{Duca et Peltonen}, ont souligné l’importance de cet indicateur pour la surveillance macroprudentielle, notamment en période de crise. Les recherches sur les mécanismes de contagion, comme les effets de liquidité et les ajustements de portefeuilles  \cite[2009]{Brunnermeier et Pedersen}, ont enrichi la compréhension des dynamiques de propagation des tensions. Ces contributions montrent que l’intégration des approches économétriques avancées dans l’analyse des stress systémiques est essentielle pour saisir les interactions complexes entre les marchés et orienter les politiques de stabilisation financière.\\

\textbf{Ainsi, les méthodes économétriques non linéaires peuvent-elles être utilisées pour modéliser et interpréter les interactions dynamiques entre les sous-indicateurs de stress systémique entre le marché actions, des changes et interbancaire, tout en intégrant les effets de régimes et les asymétries qui caractérisent les relations de stress entre ces sous-indicateurs ? Sur données mensuelles de 2005 à 2023.}\\

Cette problématique s’inscrit dans un cadre méthodologique avancé, visant à dépasser les limites des approches économétriques linéaires traditionnelles. En effet, les dynamiques systémiques des marchés financiers présentent des comportements complexes, notamment des transitions abruptes entre régimes (par exemple, entre périodes de faible et de forte volatilité) et des asymétries dans les réponses aux chocs financiers. Ces caractéristiques rendent les modèles non linéaires particulièrement adaptés pour capturer les structures sous-jacentes des interactions de stress sur les segments financiers. Ce projet vise donc à explorer ces dynamiques à travers la modélisation non-linéaire multivariée et univariée des sous-indicateurs de stress systémique pour la période 2005-2023 (données mensuelles). Deux approches économétriques complémentaires sont mises en œuvre pour répondre à la question posée.\\

Face à ces défis, la littérature en économétrie et en modélisation du stress systémique a apporté des premières solutions méthodologiques. D'une part, les modèles Markov Switching VAR (MS-VAR) proposés initialement par \cite[1989]{Hamilton}, ces modèles permettent de capturer les transitions entre régimes, en identifiant des états latents caractérisant les périodes de faible et de forte volatilité. Ils se révèlent particulièrement efficaces pour analyser les séries temporelles présentant des ruptures structurelles ou des comportements non stationnaires. Par exemple, \cite[2012]{Ang et Timmermann} ont montré leur pertinence dans l’étude des dynamiques financières cycliques. Dans le cadre de ce projet, ces modèles sont utilisés pour analyser les interactions dynamiques entre les sous-indicateurs, tout en tenant compte des changements structurels au fil du temps. D'autre part, les modèles NARDL (Non-linear Autoregressive Distributed Lag) : Introduits par \cite[2014]{Shin}, ces modèles se concentrent sur les dynamiques univariées, en intégrant des asymétries dans les réponses aux chocs. Ils permettent de distinguer les effets des variations positives et négatives sur une variable cible, une approche particulièrement pertinente pour les sous-indicateurs de stress financier, où les réponses aux tensions ne sont pas toujours symétriques. Leur application à la macroéconomie monétaire a été mise en avant par \cite[2013]{Balcilar et Ozdemir}, qui ont montré leur utilité pour modéliser les comportements asymétriques des marchés.\\

L’intégration de ces deux cadres méthodologiques permet de mieux capter la manière dont les tensions systémiques émergent, évoluent et se propagent au sein du système financier. Une attention particulière est portée à l’identification des mécanismes de contagion. Ces mécanismes sont modélisés en tenant compte des effets de régime et des asymétries observées dans les données.\\

Le projet ambitionne également d’éclairer la manière dont les autorités monétaires et macroprudentielles peuvent utiliser ces outils pour surveiller et stabiliser le système financier. En offrant des indicateurs précis et dynamiques, il devient possible de détecter les signaux précoces de crises systémiques et d’adopter des mesures préventives adaptées. Par exemple, une augmentation simultanée des tensions sur les marchés interbancaire, des changes et des actions pourrait indiquer un risque accru de propagation des tensions à l’ensemble du système financier.\\

L’analyse se concentre sur les données mensuelles couvrant la période 2005-2023, une période marquée par des crises financières majeures. Après une présentation des sous-indicateurs et des segments financiers étudiés, la méthodologie adoptée est détaillée, en insistant sur les spécificités des modèles MS-VAR et NARDL. Les résultats obtenus permettront d’identifier les périodes de stress systémique, d’évaluer l’intensité des interconnexions entre marchés, et de proposer des recommandations pour renforcer la résilience du système financier  pour le régulateur.\\

Ce projet s’inscrit dans une démarche visant à enrichir la compréhension des dynamiques non linéaires des marchés financiers. En combinant des approches économétriques avancées et des données empiriques riches, il est attendu que ces travaux contribuent significativement à l’étude des stress systémiques, tout en offrant des outils robustes pour guider les politiques monétaires et macroprudentielles.

\end{sloppypar}